\documentclass[a5paper]{book}

\usepackage[utf8]{inputenc}
\usepackage[german,english,russian]{babel}
\usepackage{hyperref}
\hypersetup{unicode=true}

\title{Стойка ЧПУ KOSY}
\author{перевод Dmitry Ponyatov <dponyatov@gmail.com>}

\begin{document}

\maketitle

Кривой перевод оригинальной документации из поставки учебного комплекта станков
WABECO~--- файлы \verb@nccad7.chm@ и \verb@ZSE3Help.chm@ с немецкого.
Комплектность поставки документации от ecoinvent хромает, могли бы хотя бы
английский вариант положить как опцию.

\tableofcontents

\part{NCCAD7}

Inhaltsverzeichnis

\bigskip

Hilfethemen für nccad7 - 18. 02. 2005

\bigskip

\chapter{Inhaltsverzeichnis} 
\chapter{Suchen von Worten} 
\chapter{Das Indexregister} 
\chapter{Grundlagen der Bedienung} 
	\section{Kurzanleitung}
	\section{Bedienprinzip}
	\section{Drucken}
	\section{Konfiguration von nccad7}
	\section{Vorlagen}
	\section{Das Menü}
		\subsection{Ansicht}
		\subsection{Datei}
		\subsection{Hilfe}
		\subsection{Maschine}
		\subsection{Parameter}
		\subsection{Simulation}
	\section{Die Icons}
		\subsection{Bearbeitung}
			\subsubsection{Drehen}
			\subsubsection{Eigenschaften ändern}
			\subsubsection{Konstruktionspunkt verschieben}
			\subsubsection{Kopieren}
			\subsubsection{Korrektur Texte}
			\subsubsection{Kreisförmig anordnen}
			\subsubsection{Löschen}
			\subsubsection{Löschen Letztes}
			\subsubsection{Rückgängig Letztes}
			\subsubsection{Skalieren}
			\subsubsection{Spiegeln horizontal}
			\subsubsection{Spiegeln horizontal mit Kopie}
			\subsubsection{Spiegeln vertikal}
			\subsubsection{Spiegeln vertikal mit Kopie}
			\subsubsection{Verschieben}
			\subsubsection{Zoom Maßstab}
		\subsection{CAD - 3D}
			\subsubsection{Ansicht dreidimensional}
			\subsubsection{Plastische Zone Kreis}
			\subsubsection{Plastische Zone Rechteck}
			\subsubsection{Plastische Zone in STL wandeln}
			\subsubsection{Schnitt bearbeiten}
			\subsubsection{Schnitt neu}
		\subsection{CAD - Besonderes}
			\subsubsection{Ellipse}
			\subsubsection{Freihand}
			\subsubsection{Kurve Approximation}
			\subsubsection{Kurve Interpolation}
			\subsubsection{Mathematische Funktion}
			\subsubsection{Outline - Generierung}
			\subsubsection{Pad/Bahn - Generierung}
			\subsubsection{Schraffieren}
			\subsubsection{Tangente}
			\subsubsection{Tangente außen}
			\subsubsection{Tangente innen}
			\subsubsection{Zahnrad außenverzahnt}
			\subsubsection{Zahnrad innenverzahnt}
			\subsubsection{Zahnstange}
		\subsection{CAD - Standard}
			\subsubsection{Bogen} 
			\subsubsection{Gerade}
			\subsubsection{Gravurtext MAX/einzeilig} 
			\subsubsection{Gravurtext MAX/mehrzeilig}
			\subsubsection{Gravurtext TrueType} 
			\subsubsection{Kreis} 
			\subsubsection{Langloch} 
			\subsubsection{Polygon} 
			\subsubsection{Polygon-Generierung} 
			\subsubsection{Punkt} 
			\subsubsection{Rechteck} 
		\subsection{CAM - Standard}
			\subsubsection{Ausspannposition} 
			\subsubsection{Bahnkorrektur} 
			\subsubsection{Insel auflösen}
			\subsubsection{Insel zuweisen} 
			\subsubsection{Kontur auflösen}
			\subsubsection{Kontur generieren}
			\subsubsection{Leitkontur} 
			\subsubsection{Tasche fräsen} 
			\subsubsection{Technologie} 
			\subsubsection{Werkstück - Befestigung} 
			\subsubsection{Werkstück - Nullpunkt} 
		\subsection{Darstellung}
			\subsubsection{Ansicht Letzte} 
			\subsubsection{Ausschnitt verschieben} 
			\subsubsection{Ausschnitt wählen} 
			\subsubsection{Neu darstellen} 
			\subsubsection{Tischdarstellung} 
		\subsection{Dokumentation}
			\subsubsection{Bemaßung horizontal} 
			\subsubsection{Bemaßung Radius} 
			\subsubsection{Bemaßung schräg} 
			\subsubsection{Bemaßung vertikal} 
			\subsubsection{Bemaßung Winkel} 
			\subsubsection{Beschriftung TrueType} 
			\subsubsection{Beschriftung MAX/einzeilig} 
			\subsubsection{Beschriftung MAX/mehrzeilig}
			\subsubsection{Bearbeiter} 
			\subsubsection{Bearbeiter letzte Änderung} 
			\subsubsection{Dateiname} 
			\subsubsection{Datum letzte Änderung} 
			\subsubsection{Datum aktuell} 
			\subsubsection{Datum Ausdruck}
			\subsubsection{Datum erstellt} 
		\subsection{Einstellungen}
			\subsubsection{Bezugspunkt} 
			\subsubsection{Fang} 
			\subsubsection{Layer (Zeichnungslage)} 
			\subsubsection{Lineal} 
			\subsubsection{Linien} 
			\subsubsection{Raster} 
		\subsection{Information}
			\subsubsection{Messen} 
			\subsubsection{Zeichnungsteil-Informationen} 
		\subsection{Symbole}
			\subsubsection{Symbol auflösen}
			\subsubsection{Symbol laden} 
			\subsubsection{Symbol speichern} 
		\subsection{Umwandlung}		
			\subsubsection{Trimmen} 
			\subsubsection{Verlängern} 
			\subsubsection{Trimmen-Verlängern} 
			\subsubsection{Trimmen 2 Teile} 
			\subsubsection{Verlängern 2 Teile} 
			\subsubsection{Auftrennen} 
			\subsubsection{Autom. Trimmen-Verl. (Kontur verfolgen)} 
			\subsubsection{Verdünnen} 
			\subsubsection{Polygon-Generierung} 
			\subsubsection{Konvertieren in Gerade} 
			\subsubsection{Konvertieren in Polygon}
			\subsubsection{Runden} 
			\subsubsection{Runden selektiv} 
			\subsubsection{Fasen} 
			\subsubsection{Fasen selektiv} 
						
\chapter{Grundlagen CAD}
	\section{CAD - Konstruktionsgrundlagen}
	\section{CAD - Kurzanleitung}
	\section{CAD - 3D Funktionen}
	\section{CAD - Technisches Zeichnen}
		\subsection{Dreiseiten-Darstellung}
		\subsection{Räumliche Darstellung}
		\subsection{Arbeiten mit Symbolen}
	\section{Grafik}
		\subsection{Bilder importieren}
		\subsection{Formulare}
	 
\chapter{CNC-Fräsmaschinen} 
	\section{Inbetriebnahme, Erste Schritte}
	\section{Handsteuerung}
	\section{CNC-Fräsen}
	\section{Teach In - Programmierung} 
	\section{Simulation mit OpenGL} 
	\section{Werkzeug-Korrektur} 
	\section{CAD/CAM-Fräsen} 
		\subsection{Einführungsbeispiel, Prinzip} 
		\subsection{Technologie-Angaben} 
	\section{3D-Fräsen}
		\subsection{Körper aus Rippen und Spanten} 
		\subsection{Plastische Zonen} 
		\subsection{Randzonen} 
		\subsection{STL Grundlagen} 
		\subsection{STL Ebenenbearbeitung} 
		\subsection{STL 4-Achs-Bearbeitung}
	\section{Bearbeitungseinheiten} 
		\subsection{Universal (Metabo)} 
		\subsection{Schnellfrequenz} 
		\subsection{Drehstrom} 	 
	\section{Arbeitshinweise}
		\subsection{WNP verschieben} 
		\subsection{Maschine aufrüsten}
	\section{Hilfsmittel} 
		\subsection{Pratze, Treppenbock} 
		\subsection{Anschlagwinkel} 		 
	\section{Spezialanwendungen} 
		\subsection{Mit Sonderwerkzeugen}
			\subsubsection{Gewindefräser ZBGF} 
		\subsection{Gravuren} 
			\subsubsection{MAX-Schriften} 
			\subsubsection{TrueType-Schriften} 
			\subsubsection{Fortlaufende Zahlen}
			\subsubsection{am Bogen}
		\subsection{Leiterplatten} 
			\subsubsection{Mit nccad entwerfen und fräsen} 
			\subsubsection{Von Layout-Programmen bearbeiten} 
			\subsubsection{Von Layout-Programmen bohren} 
		\subsection{Zahnräder} 
			\subsubsection{Grundlagen} 
			\subsubsection{Bearbeiten} 
			\subsubsection{Praxis} 			
		\subsection{Schneiden} 
			\subsubsection{Schleppmesser} 
		 
\chapter{CNC-Bohrmaschinen} 
	\section{Bediengrundlagen}

\chapter{CNC-Drehmaschinen}
	\section{Daten und Fakten} 
	\section{Inbetriebnahme, Erste Schritte} 
	\section{CNC-Drehen} 
	\section{Simulation mit OpenGL} 
	\section{Werkzeugverwaltung} 
	\section{Testhilfen} 
	\section{CAD/CAM-Drehen} 
		\subsection{Prinzip} 
		\subsection{Mehrfachzyklen} 
	\section{Spezialanwendungen} 
		\subsection{Gewinde-Drehen} 
		\subsection{CNC-Zyklen} 
 
\chapter{Dosier-Systeme}
	\section{Grundlagen} 
	\section{Schubdosierung} 
 
\chapter{Automatisierungs-Systeme}
	\section{Grundlagen} 
	\section{Schaltfunktionen während Bewegung} 
 
\chapter{KOSY-Steuerungen} 
	\section{Standard-Ausführung} 
	\section{Wabeco-Ausführung} 

\chapter{Spezial-Funktionen/Programme} 
	\section{Hilfsprogramme} 
		\subsection{Zeichensatz-Editor} 

\chapter{Import/Export} 
	\section{CNC-Programmexport} 
	\section{Postprozessoranpassung} 
	\section{Schnittstellen} 
	\section{2D-Import} 
		\subsection{DXF}
		\subsection{HPGL} 
		\subsection{Nachbearbeitung} 
		\subsection{Scannen und Vektorisieren} 
	\section{3D-Import} 
		\subsection{STL}
	\section{3D-Export} 
		\subsection{STL}

\chapter{Optionen und deren Bedienung} 
	\section{Abtasten}
	\section{Drehen} 
	\section{KOSY Wagen} 
	\section{Mindermengendosierung} 
	\section{Schubdosierung} 
	\section{Spooler} 
	\section{Tiefenregelung} 
	\section{Werkzeuglängenmessung} 
	\section{Werkzeugwechsel} 

\chapter{Umgang mit dem System}
	\section{Service und Wartung} 
		\subsection{Fehlerliste} 
		\subsection{Hotline} 
		\subsection{Massiv-Körper reparieren} 
	\section{Transport} 
		\subsection{Massive Maschinen} 
		\subsection{Paketversand} 
 
\chapter{Anhang}
	\section{Liesmich/Installation} 
	\section{Anschlussbelegung} 
	\section{CAM-Technologien} 
	\section{Dreh-Zyklen} 
	\section{NC-Befehle} 
	\section{NC-Kurzbefehlsliste} 
	\section{Netzwerk-Installation} 
	\section{Tastenbelegung (HotKeys)} 
 
 
\part{ZSE3}

\end{document}
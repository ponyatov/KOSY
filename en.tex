\documentclass{book}
\usepackage[a4paper,left=20mm,right=10mm,top=10mm,bottom=10mm]{geometry}

\usepackage{graphicx}

\newcommand{\nccad}{\textbf{NCCAD}}

\newcommand{\email}[1]{$<$#1$>$}

\newcommand{\fig}[2]{
\bigskip
\includegraphics{#1} \\ \textbf{#2}
\bigskip
}

\title{NCCAD7: Help KOSY3 21-11-2006}
\author{Dmitry Ponyatov \email{dponyatov@gmail.com}}

\begin{document}

\maketitle

\part{Contents}

\tableofcontents

\part{Search words}

Search words
Revised: August 2002


Supposing you want to know more about the term "Zeropoint": In our helppages the term appears on several pages. Use the function "Search" in order to find them and the relating texts. Take the following steps:

Open the index card "Suchen" ("search") by clicking on its tab. 
Enter the word you are looking for into the field "Suchbegriff(e) eingeben" ("enter the searchword/s"). 
Mind the spelling. 
Small/capital letters make no difference in the search function, e.g. you can enter "Manual operation" or "manual operation". 
Search words you have already entered can be called up again in a pull-down-menu (arrow downwards). 
You can add logical connections to the searchword via the button "Arrow to the right", beside the entering field. 
Click at "Themen auflisten" ("list of topics "). It appears a list of all pages where the search word is found at least once. 
Select one of the pages by double clicking. The found words are marked. You can find them by scrolling. 
The following picture gives you an example of the different stages of the search:

\part{Index register}

The index register
Revised: August 2002



If you want to know more about a certain topic or application, you can find all kinds of keywords in the index register. They are sorted according to the topics.


How to use the index register

Enter the word into the field "Look for keyword". Each letter you enter leads you to a corresponding subject in the index window:



When you have found the wanted term, click at it twice. The corresponding helppage appears in the window on the right. 


\part{Basics of nccad}

nccad is a CAD/CAM-Software, 
with direct machine control

Menu of the CAD/CAM-Software nccad 
 

Open the branch and select your topic! 

\chapter{User surface}

The user surface of nccad
Revised: October 2005

nccad is a CAD/CAM-software, leading to automation and machining through a relating mechanics (e.g. CNC-machine).


1. The menu of our CAD/CAM-software nccad 



2. Make your choice in the menu File 
In most cases of nccad, a controlling program for some sort of automation is loaded or generated which is later executed by the machine. Before starting to work, you have to decide in the menu File...:

...if you want to generate a new program, 
...if you want to open an existing CAD/CAM-drawing or
...if you want to open one of the last programs you worked on.

There a always two ways of programming:

CNC-programming with codesentences after DIN/ISO 66025. 
CAD/CAM-programming, starting from a drawing, differenciated according to the type of machine: 
for CNC-milling machines and Coordinate Systems (KOSY = Coordinate System) 
for CNC-Lathes 
for special applications. 
 

Therefore you have to decide in the menu File:

if an existing CNC-program shall be opened or newly generated; 
if an existing CAD/CAM-drawing shall be opened or newly generated and which type of machine it is for. 
For CNC-Programming to mill and for Coordinate table applications please get more information about the steps to take under Basics Milling machine/CNC-Programming...

For CAD/CAM-Programming click at CAD/CAM milling -new drawing in the menu File (see picture above) and 
go on reading under 3. to learn about the user surface. 

You will find more details for CNC-miling machines and Coordinate Systems under Basics CAD/Basics of construction, under CNC-milling machine/CAD/CAM milling/Beginner's example, as well as in CAM-Technologies for machining.

You will find more details for CNC-Lathes in Basics CAD/Basics of construction, and under CNC-Lathe/CAD/CAM turning.

 

 
3. The CAD/CAM screen for applications of the coordinate table (Milling, dispensing, automating, etc.)
After having clicked CAD/CAM Millilng-new drawing in the menu File, the CAD/CAM-start screen opens. It shows different areas. 
The status bar at the bottom is of special importance: There the Software tells you what you are supposed to do, which information you have to give, what you have to enter next, which function is activated etc. 
For going on the Icon-Bar on the left is important. 

 

  

4. The Icon-Groups 
For a better clarity the icons and with them the different CAD/CAM-functions of nccad are divided into groups. Each group has a button with a heading. If you click at such a button, the relating icons show up and you can use them. Another click closes the group again.
This method has the advantage that you can limit the space for the icons: You just open the groups you need.
At the beginning, none of the icons is activated, therefore the status bar tells you: No function activated. Please choose.......(see picture above).

 


5. Choose the position of the icon bar freely
At the top of the icon bar there is a blue caption. You can take the bar and drag it into a new position by setting the mousepointer in the caption, pressing the left mouse button and moving the mouse. Let go the left mouse button at the position you want.


 


6. Dock the icon bar at the outer edge 
You can dock the icon bar to the left or right side of the nccad-window frame: Move the mousepointer into the blue caption, press the left mouse button, keep it pressed and drag it to the frame until a small dotted line appears (middle of the icon bar has reached the frame). Let the mouse button go.

 


7. Display of the coordinates and layers
The coordinates are displayed at the bottom of the drawing area. Move the mouse to one of the fields, wait a moment and you are explained the meaning by a tooltip text. You will find details about entering coordinates under Basics CAD/Basics of construction/ Entering coordinates.
Beside the display of the coordinates you will find information about the layers: Layer-N�, colour of layer and the symbol for the layer protection (is this layer protected against unintentional editing yes/no). You will find details under Basics CAD/Basics of Construction/Layer.

 


8. Choose a function in the icon bar
At the beginning, none of the icons is activated, therefore the status bar tells you: No function activated. Please choose.......(see bottom of the picture). Therefore moving the mouse within the drawing field has no effects. Only after clicking one of the icons of e.g. the group CAD-Standard, the status bar gives you the corresponding information and you can start constructing. 
You will find more details under Basics CAD/Basics of Construction and in all other help topics .

You will find a training exercise "From drawing to machining a workpiece" under Teachprogramm CAD/CAM.

\chapter{CAD -- Short instruction}

Short instruction for the CAD-part of nccad
Revised: April 2006

Status bar and Icon bar
Input of Coordinates
Tricks to help drawing
Right Mouse button 
Layer 


Starting via the Menu<File><CAD/CAM New drawing>
It appears the window you see above, with a drawing field which is slightly bigger than a DIN A4-paper (corresponds to the Y-table of the CNC-machine Standard A4).
The different window elements are marked in red on the illustration .


Status bar
Always pay attention to the status bar right on the bottom. It tells you, which nccad function is active and which step you are supposed to take next, e.g. after starting: "Choose a function in the menu <File>, please " and after you have chosen <CAD/CAM New Drawing>: "Please choose a drawing function".
Icon bar
The icons are combined in groups. Clicking a group heading opens or closes the group.

Input of Coordinates with the crosshairs
Example: Open<CAD Standard>and choose the icon LINE. The status bar tells you to "Choose starting point". Move the mouse on the drawing field and watch the coordinate display below. Confirm the wanted position by pressing the left mouse key (click).
Now the status bar tells you to "Choose final point". Move the mouse and again watch the coordinate display, especially the field CA. It shows the distance to the starting point. Click at a wanted position, the line is constructed.
Input of Coordinates by keyboard
- Absolute Coordinates (CA = Distance to the zeropoint of the drawing resp. workpiece): Key <K> or click into the field CA.
- Relative Coordinates (CR = Distance to the last coordinate point): Key <K> or click into field CR.
Then the position of the coordinates are entered: Value X, e.g.: 123.45> comma> Value Y, e.g.: 33.9> Enter.
The display shows then: 123.45,33.9
The decimals are separated by point. X- and Y-values are separated by comma.
Advantage of this sort of input: Positions can be defined in 1/100mm.

Mathematical signs for basic calculation and brackets are allowed, e.g.: 5*22.3,2*(28+47.9/2)
Input of polar coordinates
Activate one of the entering fields for polar coordinates below the drawing area:
- Polar Absolute (PA = Distance to the zeropoint of the drawing resp. workpiece): Click twice key<P> or click field PA.
- Polar Relative (PR = Distance to the last coordinate point): Click once key<P> or click field PR.
The coordinates are entered: angle, e.g.: 45> comma> length, e.g.: 33.9> Enter. 
Input of angles
Whenever the status bar asks you to enter an angle (e.g. with edit /ROTATE), the key <W> can be pressed which allows you to enter the exact value by keyboard.

Tricks to help drawing
Orthogonal drawing (Mouse movements are restricted):
- Press the shift key while drawing>> mouse can be moved vertically only.
- Press the ctrl key >> mouse can be moved horizontally only.
- Press both, shift and ctrl keys>> mouse is fixed, does not "wobble".
Construction catch (the search window engages in a construction point nearby):
The catch is turned on/off by pressing the shortcut <Ctrl> + <F9> or by clicking at the symbol on the extreme right of the status bar.
Shortcuts /Hotkeys (for rapid operation):
You will find a list under Appendix/ Use of keys

Right Mouse button
Pressing the right mouse button in the drawingfield will show a PopUp-Menu with some important settings.

Layers (layers of the drawing and their meaning)
The choice of the layer can be done below the drawing field (on the right beside the coordinate display) by clicking on it or by <shift>+<Number>- or via the Icon-bar under settings/LAYER. There you can also decide about visibility and protection against unintentional editing. This last function is also achieved by doubleclicking the symbol of the lock in the status bar.
Drawing elements in layer 1 to 8 can be drawn in certain colours - and also be machined by the CNC-machine. Changing the layer means automatically changing the colour. The colours can be chosen in the menu View/Colour.
Drawing elements in layer 9 are for documenting only (legend, measures, auxiliary lines etc.). They are not machined by the CNC-machine. Calling up one of the icons in the group Documentation means automatically layer 9 with its colour 'black'.
Auxiliary lines, only to be drawn in layer 9 (Centre lines, construction lines etc). If you choose LINES from the icon group Settings you can also take different dotted lines which are also represented in layer 9 only.
Layer 10 with its colour 'red' is reserved for CAM-functions and CAM-representations. 

 

\chapter{Configuration of nccad}

Configuration of nccad
Revised: April 2006


Make yourself an individual version of nccad. You can:

generate new ICON-bars 
modify existing Icon-bars 
modify the numbers of functions 
reduce the menues 
influence the start features 
Please have in mind: The configuration must be made with the present version - and used with the same version. If you use it with other versions you will get error messages.

More information:


1. Generate a new ICON-bar and modify it

Take the mouse cursor into the Menu bar 
press the right mouse key, 
a PopUp-Menu, as in the picture above will appear where you can, either, hide the ICON-Menu, or 
Customize it. 
See the following information: 

1.1 Call up a new bar
After having chosen Customize,

click at New and enter a name for the new ICON-bar (see picture). 
After clicking at OK an empty ICON-bar appears which you can drag to any position. 

1.2 Fill your new bar

In the window Customize click at the tab Commands, 
in the field Categories select the ICON-Group or the menu from which you want to take the ICON to be placed into the new bar. 
Select the wanted ICON by the left mouse key and drag it into the new ICON-bar with the mousekey pressed. At the inserting place there is first a black bar (see picture), which disappears when you let go the mousekey and the new ICON is inserted. 
 


1.3 Put the new bar into its position
You can take the ICON-bar and move it to any positions. When you take it to certain places the new bar gets dotted lines (see picture) and it is "docked" there.


1.4 The new bar is at its position
The picture shows the docked at ICON-bar below the menu bar, an ideal position for private ICONs. 

1.5 Remove ICONs

The window "Adapt" has to be opened (see picture 1.). 
Then you can click at one of the ICONS (out of each Icon-bar visible on the screen) by the left mouse key and drag it into the free nccad-drawing area, it will be deleted. 
 


2. Configurate the menu
As described under 1. you first have to

press the right mousekey in the menu line - and select Customize, in order to be able to make modifications. 
As soon as the window "Customize" is visible, you can click at one of the menus in the Menu bar (see picture, example:View). When you click at one of the lines in the menu window (e.g.Colour) by the left mousekey and drag it into the free area it will be gone in the menu window. 
The button "Reset" in the window "Customize" can restore the original state.

 

3. Save your configuration, open and start it
As described under 1. and 2., you have set up your proper ICON-bars and menus. You want to have this state saved to be available next time? The following information will help you:


3.1 Save the configuration or open it
The present configuration state on the screen is saved in a file with the ending *.ncc via the menu File under Save configuration.

You can call up and use a saved configuration via Open configuration at any time.

3.2 Start your configuration automatically
You can call up a configuration together with starting nccad , e.g. through the following Windows-Command line under Start/Execute or via a desktop link:

c:\kosyx\nccadx.exe /ul=0/men:c:\kosyx\beginner.ncc

x =version number, this is normally 7

First nccad itself is called up, then follows after ...exe a blank, then the information about the configuration 
ul=0 means that the configuration can any time be modified after starting nccad .
ul=1 means that the configuration cannot be modified after starting nccad .
ul... may also be left out. 


3.3 Start your configuration and machining file automatically
You can call up a configuration together with a machining file when starting nccad, e.g. through the following Windows-command line under Start/Execute or via a desktop-link:

c:\kosyx\nccadx.exe /ul=1/men:c:\kosyx\beginner.ncc c:\Fertigung\Triangle.knc

x =version number, 

First nccad is called up, followed afetr ...exe by a blank, then follows the path and file name for the configuration and another blank, followed by path and file name for the machining file, here a CNC-program.
ul=0 means that the configuration can any time be modified after starting nccad .
ul=1 means that the configuration cannot be modified after starting nccad .
ul... may also be left out.


4. Configuration files and Start Buttons
In the main directory of nccad (see picture above) different configuration files are laid down, examples:


110_Einsteiger.ncc
(pure beginners)  This version is a very first version for pure beginners. Drawing can be compared to the conventional method. For editing functions, only DELETE and CORRECTION TEXTS are available. This configuration has a smaller range of commands than 120_Anfaenger.ncc! 
120_Anfaenger.ncc
(beginners)  Beginners need only few commands for simple drawing- and editing functions. Suitable for basic training! 
210_Fortgeschrittener.ncc
(advanced)  Advanced applicants work with functions which are used in an educational context. Machine control is possible. 
300_Alles.ncc
(all)  In this full version, all functions can be used. 
    
310_Platinenfraesen.ncc  
(Milling PCBs) The necessary functions for milling PCBs (GENERATE PAD/PATH � GENERATE OUTLINE � GENERATE POLYGON) are integrated into the Iconbar for direct selection. The range of commands is the same as in the advanced configuration 210_Fortgeschrittener.ncc. 
    
\chapter{Printing}

Printing within nccad
Revised: April 2006 

Whenever you want to print something, the following pages will help you. Please, distinguish between: 

Printing help subjects for the manual 
Printing copies of the screen 
Printing CAD/CAM-drawings and CNC-listings  
Set page for printing 
Printing is always done by the currently installed and activated printer, the control of the process of printing itself is done by Windows. 





1. Printing help topics
The help topics are basically HTML-pages. The Internet-Explorer is used to represent and manage them in the background and is also responsible for printing the individual help subjects. 

The only thing you have to do is to choose the subjects from the overview (left) and to click at "print" in the active window. The rest will be done by Windows and the current printer driver.

 



2. Printing copies of the screen 
Pictures and details of pictures that are shown on the screen cannot be sent to the printer directly. But you may want to print out certain presentations and windows, such as: 

Explanations to CAD-ICONs (F1-Help pages) 
Perspective presentations of plastic zones. There we installed an extra shift for getting economical black/white prints of the presentation. 
So, how to proceed? You use a method called "ScreenShot" and the transmission to a picture processing program, where the picture can be edited and printed. The "clipboard" of Windows, which is easy to handle, has to be used for it:

To copy the whole screen into the clipboard you press the key [print] (on the right beside the functions-keys). You can see the picture you have taken above. 
To copy the activated window into the clipboard you press the combination of keys [Alt]+[print]. On the picture above only the window with the blue top bar would be copied into the clipboard. 
To insert the copied picture into a previously started picture processing program the key combination [Ctrl]+[V] (insert) will do in general. 
Then the picture can be modified- (sized down, cut, commented, changed...) and, of course, printed, too.

 


3. Set the page for printing
The picture shows the window Set page for printing. It serves to set the printing parameters. The fields "Acticve printer" and "Page borders" are generally known. There are however special nccad-options:

3.1 Options
When the fields are activated:

Header
the name of the person dealing with the file and the date of the last modification are printed on top of the page 
Footer
the name of the file, the printing scale and the page number are printed at the bottom of the page 
Punch mark
A marking where to punch is printed 
Only black/white
The format is suitable for a black/white printer 
3.2 Printing scale CAD
There are 3 options of how to print the contents of the drawing:

 
3.2.1. Variable scale, variable detail: 
This corresponds to the only printing option which was available in the older versions.
After clicking at "Print" the first corner of the plotting window must be marked.
Then you are asked to mark the second corner of the plotting window (blue frame). At the same time you see in the status bar the scale in which the drawing is printed onto a DINA4 piece of paper. 
3.2.2. Fixed scale, fixed detail: 

After clicking at "Print" the left front corner of the plotting window must be marked. 
The plotting window (bleu frame) has the size of a DINA4 piece of paper, so only the details which are within that window can be printed.
The scale can be changed in the fields "Enter scale".


3.2.3. Fixed scale, variable detail: 
After clicking at "Print" the first corner of the plotting window must be marked.
Then you are asked to mark the second corner of the plotting window (blue frame). When printing on a DINA4 piece of paper, the marked detail is printed in the set scale. 
The scale can be changed in the fields "Enter scale". 

Please notice when printing in scale 1:1 that the represented drawing area A4 (or others) is in fact the Y-table, which is bigger than the piece of paper ( the Y-table must have room for fixing it). Leave therefore a wide margin when you know that you want to print it out 1:1 later. By the way, you will find Z-frames for DIN A4 in all symbol-libraries. So, if you load the symbol when starting  and draw within this Z-frame you can print 1:1 later.


3.3 Printing the technology-data additionally
For that, you have to activate the function drawing and technology (see picture 3). Then, automatically, the technology-data is printed out on separate sheets after the drawing has been printed. 



 



4. Print the files
There are 2 possibilities:

4.1 Print CAD/CAM 
As soon as you have made - or opened a drawing in nccad, the current state can be printed. The printer is set via File/Set page. The printing itself is started via "Print" (see picture) and is done by the printer installed in Windows.

4.2 Print CNC-programs
As soon as you have made - or opened a CNC-progam in nccad, the current state can be printed. The printer is set via File/Set page. The printing itself is started via "Print" (see picture) and is done by the printer installed in Windows.

\chapter{Templates}

How to use templates
Date: Sept. 2004

Templates are nccad-file serving as work base. The elements contained in them (like drawing parts, fonts, technologies, etc.) cannot be deleted, shifted, copied, or others. It saves time and enables you to work and edit freely.

1. Principle
If you want to save frequently needed CAD/CAM-drawings as working base, take the following actions:

Make the CAD/CAM-drawing with basic elements 
Save as Template (its extension being *.cdt)
The elements are marked and protected for future use 
Open the template at any time 
Add new elements (the template-elements are unchangeable) 
Save as CAD/CAM-file (its extension being *.cad) 
The operational functions in detail:

 


2. Template-functions
In the menu File/templates you get to a sub-menu where you can select the wanted function:


2.1 Open templates
Click on Open template to get the picture above. The content of the folder "Templates" is presented, click on the wanted template to load it.
The folder proposed can be set in the menu Parameters/Files/Select folder... and can be changed there, if wanted.

As a start,the folder contains some templates; you can see some of them in the following examples:


Example 1: Template "Milling file" with pre-set zeropoint, workpiece-changing position and filename...


Example 2: Detail of the file "Drawing frame"...

You can draw into the template, and edit the drawing (delete, shift...)without changing the template elements, which are locked.

 


2.2 Save as template
You can save each file as template by clicking at the menu File/Templates/Save as template.
For saving, the folder Templates is proposed, the files gets automatically the extension *.cdt . The folder can be changed in the menu Parameters/Files/Select folder.... 

 


2.3 Edit template
The template can be modified, completed or reduced. Click at the menu File/Templates/Edit template, the protection of the drawing elements is cancelled.

The file can be edited like any CAD/CAM-file - and it can be saved afterwards, either....

... as a normal CAD/CAM-file 
... or again as a template, see 2.2.

\chapter{The Menu}

The menu bar

This is the menu bar. For more information, click a keyword further down in the table of contents.

\section{View}

Menu view
Revised: December 2004

Here are elements available to modify the way of representing a drawing.



Colour
Layer 1 to 8 are displayed in 8 different colours (except for red and black). This way each user is able to make up his own "range of colours". The colour switches automatically when you change the layer by the icon "LAYER" or by the short keys SHIFT+F... 
The colour of the current layer is shown for information. 
The colour settings for the table is of importance if the contents of the picture is to be copied via <Ctrl> +<C> into a DTP-program.

Drawing area
You can choose the size of your drawing area. The possible size is selected by mouse click and registered in the program by clicking on "save". 
The default value is DIN A4. 
Coordinates display
The coordinates are displayed in cartesian and polar, in absolute as well as relative values. An additional setting window appears. Here you are able to hide display information to simplify and reduce the size of the window.
The settings made here can be saved under Save configuration in the Menu File.

Helplines
After clicking this menu dotted construction helplines are inserted in red. 
For example additional polygons for curve constructions or textframes for inscriptions. 

\section{File}

Menu file
Revised: October 2004

Each programming activity with nccad starts by selecting the file to be processed. 
Either you make a new file, or use an existing one. The functions "SAVE" and "PRINT" are started here too. 
The proper exit out of nccad is also done in this menu. 


Generally, you have to decide whether to program the CNC-machine in the CNC-way or by the CAD/CAM method. 



Please mind the following :

CAD/CAM New drawing
The way from the drawing to the finished workpiece starts here. 
In case another file is already in work you are asked if you want to save it . Then the screen is cleared and the drawing area is displayed, representing the Y-table. Simultaneously, the memory for the drawing pieces is cleared. Choose the functions for drawing in the icon bar. 
CAD/CAM New special drawing
Here you find the special CAD-functions. A sub-menu appears from which you can choose. 
Turning
If you own the turning equipment, these software functions become accessible. 
You can find more information in our Helptopics. 
Model building 
Particularly for model builders who want to construct rib-shaped bodies (like plane wings or boat bodies) 
Its operation is explained in the help topics 3D/Model building. 
CAD/CAM - Open drawing
You want to load an existing drawing. In case another file is already in work you are asked if you want to save it. The screen is cleared and a window is opened which allows you to choose between the CAD-files in the announced directory. 
CNC - New program
Start a new NC-program.In case another file is already in work you are asked if you want to save it. The screen is cleared and you see an empty edit window. Simultaneously, the memory for the NC-sentences is deleted. 
CNC - Open program
Via the explorer window a KOSY CNC-file with the extension *.KNC is chosen and loaded. The text editor is opened.
In case another file is already in work, you are asked if you want to save it. The screen is cleared and a window with the available CNC-programs is opened. Select a file. 
Import file
Here you find the special load functions. A sub-menu appears from which you can choose. 
DXF-file
If drawings are made in other CAD programs and saved as DXF-files, they can 
be loaded here, and they have to be given their technology data in nccad. 
Imported drawings are captured form layer 0 to 32000, layer 0 is moved on to layer 1 and layers bigger than 9 appear in layers 1 to 9. Colors are only checked with points. Drawings which are too large are automatically reduced to table size in whole number scale. With very small drawings the program asks if the scale shall be adjusted to table size or not. The graphic simulation and the CNC machining is however done in its original size, if you have not changed the scale in <Parameter>. See alo under Import/Export 2D 
HPGL-file
Some CAD-programs use plotterfiles (HPGL-files). If the CNC-machine has to replace the plotter, this import possibility is an advantage. 
The technology data of nccad offer the possibility for contour correction and of milling in more than one steps. See also under Import/Export 2D 

Surface-file
If you have scanned a plastic body and saved the scanning (*.txt), you can load it via this function. A plastic zone appears on the screen. 
STL-file
If 3D-bodies are drawn in another CAD program and saved as STL-file (*.stl), they can be imported here. More information in our help topics under CAD 3D/STL. 
Picture-file
Here you can import picture-files in the formats BMP, GIF and JPG. You will find more information under Basics CAD/Graphics...... 
Last file(s)
You get a list of the last files you worked on. 
Save file
Your work is saved in the announced directory. If your project is not yet given a name (the top of the menu bar shows "NoName" as file name) you are first asked to enter the date and name of the worker, then you can enter a file name. 
Save file as 
A window is opened to enter the date and name of the worker, then to enter a file name. So you are able to save existing files under a new name if they were modified, for example. The file is saved under the new name in the announced directory. But you are able to change the directory in the window, too. 
Export file(s)
Here you find special functions of saving. A sub-menu appears from which you can choose. 
Model-building (3D -> 2D)
If you have made several rib-shaped profiles in "CAD Special - 3D" they are now projected on a surface by this export and can then be further processed as 2D-file (e.g. give them technology and mill them).
HPGL-file
The drawing is saved as a plot-file (* .PLT). 
You use this file to print it out with a plotter or you import it to a DTP-program. 
If you make drawings for HPGL-export, don't use true-type-characters (inscription, engraving-texts), they are not displayed. 
KOSY list
When a drawing is ready and completed with technology data, you normally start machining or graphic simulation via <Machine> or >Simulation>. In each case the postprocessor is activated generating a controlling list for the CNC-machine. This list can be saved separately. It has the same name as the drawing-file but gets the file-extension .CNC. 

Set page 
Here you make settings how the pages are to be printed. The window is easy to be understood and explains itself. You will find more information under Basics of operation/Printing... 
Print
The current file is printed. 
You find the printer settings in <Set page> (see above). 
Configuration
You will find more information in detail under Basics of operation/Configuration... 
Templates
You will find more information in detail under Basics of operation/Templates


nccad7end
The menu function "nccad exit" is the best way to close nccad.



\section{Help}

Menu help
Revised: December 2004

Here you find information about our Help topics concerning the different system components and their operation as well as information about the present version.




Help Topics...
You get an overview on the helptopics and a so-called searchtree to choose your subject under the condition that the MS-Internet browser (since version 4) is installed on your PC.
In the printed part of our manual you get information on installation and handling of this help system.
In addition to that: 
Search
You can look for search words by entering the term you are looking for. A list of topics appears where this term is applied. The topics can be chosen and clicked at. Another possibility is to look in the index register.

Print
You can print all chapters of the help topics. A colour printer is useful. More information under Operation/Printing...

File Information
Here you are informed on size and type of the file, important for service. 
Version
You are asked for your version in service cases. 
Hotline
Before calling the hotline you should read these lines to spare costs and misunderstandings. 
Licence
You may be asked for your licence in service cases. 


\section{Machine}

Menu machine
Revised:December 2004



Since version 7 we differenciate between different types of machines. You have to activate the CNC-Lathe in the Menu Parameters/Machine/KOSY first to be able to select it here.

Please mind the following:

CNC-Machine
The manual operation window appears ("manual operation"). Out of security reasons this window is exclusive on the screen, i.e. you can quit this window only by the button "CNC-machine end".
The manual operation controls the following functions: 
- Machine-control by hand 
- Basic settings of the machine 
- Input of NC direct commands
- Administration of the zeropoints 
- Display of the mode of the machine 
- Display of the positions 

For more information press the function key [F1] while the "manual operation" window is opened or open Basics of operation/Manual operation resp. Lathe Basics/first steps


Drilling machine
The CNC machine can be used as "intelligent drilling machine" when you connect a joystick to the game port. 

\section{Parameter}

Menu Parameters
Revised: October 2006

Here you have elements concerning the basic settings of your system components.



1. The different functions of the menus
There are different sub-menus for different groups. Please mind the following:

Machine 
The way of machining and the typical data of the CNC-machine connected with the PC are set here.

Parameter...
This function allows to set all system parameters, e.g. the PC-port, where the CNC-Machine is connected to. The speed factors are set here, too, and other functions...
Since nccad7.5 we have the possibility to hand over the parameters-file together with the installation and to have this file updated later by your supplier (load). When OEM-customers control a very specific machine, a parameters-file must be given.
In each case, the very complex and sometimes complicated parameter settings are to be entered by specialists only !!! 
Check parameters
Here you can see which parameters are set. 
Load parameters
A parameter-file is updated, e.g. by an update of your supplier 
Edit parameters
Here you get a warning first: Modifications can be dangerous......
You can change all settings. 
Extras... 
Workpiece Copies:The machining is repeated on the workpiece surface as often as possible (copied).
An example: number of copies X=2, Y=1 and shift X = 5, Y = 10. This means: Including the original, the machining is done 3-times in X-direction and 2-times in Y-direction, in total number 16 times on the whole surface. 
The shift must be at least as big as the maximal way of machining in the relating axis. 
Counter: If you want to engrave signs with consecutive numbers for example, use this function. The start-number tells the first number to engrave, each following engraving adds the step width.
The relating texts and text-formats are given as engraving-text, for example: 
CNC-machine number: * / 000.00 / * 
More details under Special applications/Engraving/...subsequent numbers.

Options: The test mode enables you to do a step by step machining. It can be used with the graphic simulation as well as with the working of the CNC-machine. 
To activate/deactivate, click the field "Testmode". 
The control of the single step functions is done via the functionkeys or the buttons. 
Automatic Home-run: If this field is activated, the machine zero point is approached automatically when the machine is started. So you get an exact zero position of the workpiece. 
Specification: This window is used with customer specific sytems only.
Here you can choose the existing and connected system components.They have an influence on the machining-data. 
Type of the CNC-machine... There is a default setting out of the present delivery program. 
Factor... ..it depends on the strength and the stability of the CNC-machine and is declared by the manufacturer. When in doubt, edit a value lower 1. 
Spindle... ..the machining unit has limits for its rpm. Its values are to be entered or already default.

Simulation 
Workpiece-Zeropoint... 
Detail... 
Speed
Since nccad6 the speed of the simulation can be adapted to the clock frequency of the PC, so that, for instance, a true-time simulation is approached. Please try different settings.
Since nccad7 there is a differenciation between the speed setting for Standard Simulations and OpenGL-Simulations. 

CAD 
PCB
The size of PADs and the width of the circuits are assigned here to the layers 2 to 8.
There are default values given which can be edited individually. See also under Special applications/PCB..

Documentation
To set and save default values. The height of the fonts (in pixel or mm) include ascenders and descenders. 
CAM 
Milling Scale
The milling scale shows the relation between "size of the drawing and size of machining". With the setting 2:1 it is possible, for example, to mill a drawing in half the size of the drawing. Normally the default scale is 1:1. 
If the scale is other than 1:1, measurements and ruler are given in natural values, the display of the coordinates in the status bar, however, shows the drawing units. Via ZOOM Scale the drawing can be adapted to the milling scale. 
Technology
The default values for the technology-window can be set here.
E.g. always activate a relay for the machining unit (BAE) and set a value for the feed.

Tools
Is of importance 
if an automatic tool changer is activated with Turning or Milling, 
if you work with a correction memory when Turning or Milling, 
if OpenGL-Simulation is used with Turning and Milling. 
For more information see below

Files 
Choose folders...
You set the folders for .... 
- drawing files
- NC-Program-files
- KOSY-lists
- DXF-files
- tools
- symbols 
As soon as you call up the functions <File> <Load> or <Save>, the folders you have set here are used.
 

Import/Export
Here the change metre/inch is done.
In "Export" additional saving of the KOSY-list can be activated. It is then saved together with the drawing files. It is available immediately after opening so you need not spend time in transmitting it before machining (as long as no modifications have been made).
 

2. Tools and their Parameters
They are worth to be paid special attention:


2.1 Call up the Tool management and general remarks

For Computer Assisted Manufacturing (CAM) you need different tools. They have to be given to the Software-System in some way and their data have to be managed. This is done in the menu Parameter/CAM under 3 different tool keywords:

2.1.1 The tool memory for machining
In the tool memory, the correction values are saved which are needed for the shift of the workpiece zeropoint (WZP) in case of a tool-change. It is therefore called Tool-Correction memory and is of importance for the actual machining only.

The Workpiece-Zeropoint has to be shifted after the tool-change, if the tool has other dimensions, i.e. if it hits the workpiece at another spot (e.g. it looks more out of the clamps, the turning tool has a side shift in its holder, etc.). 

The Shift of the WZP can be automated, when...

.... the Tool-Correction memory is activated. Menu Parameters/Machine/KOSY/Options. 
.... the tools have been measured and the correction values have been entered into the table of the tool memory. This table is useful with CNC-milling machines and it is particularly important and critical for tools for the CNC-Lathe. 
.... the new tool is entered after manual tool change (in manual control press the keys <Ctrl>+<n>, n = number of tool) or 
... the automatic tool changer is used with CNC-milling machines equipped with a tool changer or with CNC-Lathes with tool-changer. 
For a better overview, please pay attention that the Numbers of the tools in the tool-correction memory and the numbers of the tools in the tool-magazine are meant for the same tools. At present there is no automatic accordance.



2.1.2 The description of the tool 
This window allows to enter the geometry of the tool to be given to the Software. It explains itself.
These geometry data are neede for the OpenGL-Simulation and after nccad -development also for contour correction.

Please note the difference between tools for milling and turning (index-tab on the left).

For everyday use it is helpful to give the tools meaningful names, which tell the essential data and its application, e.g.: ShankMill3_2mm (Cylindric mill, 3mm shank, 2mm diameter). The name of the tool and the file name can be largely identical, as long as no symbol is used which is forbidden with file names. Little trick: Mark the name and copy it into the clipboard by <Ctrl>+<C>, you can then insert it by <Ctrl>+<V> into the explorer-window under the file name. 

The Geometry of a tool is saved in a separate subdirectory Tools . See also chapter 2.2.


2.1.3 Tool-magazines
They are helpful for editing a tool set of the most frequently used tools. It helps to avoid permanent looking for tool files.

You can make a tool magazine respectively modify it, by.....

selecting one of the indextabs (milling or turning), 
marking the number of the tool in the table, 
choosing 'load tool', to select one of the existing tool files,
or 'delete tool', to take a tool out of the magazine, 
choosing 'Save magazine', to save the table on the hard disk.
At present there is only one magazine for milling and one for turning, saved under the file name Default.MGZ . 
The magazine helps you to find the tools quickly and - in case of a manual or automatic tool change- to assign the number of the tool to the tool data and the tool geometry. At present, this is true for simulation only, later it will also be possible for machining.

For a better overview, please pay attention that the Numbers of the tools in the tool-correction memory and the numbers of the tools in the tool-magazine are meant for the same tools. At present there is no automatic accordance.

2.2 Tool files
They are saved in a separate sub-directory:

C (hard disk) 
Kosy7 (Home directory) 
Tools (Tool-subdirectory) 
Milling (milling) 
*.TOL (tool files, geometry-data)
Default.MGZ (default-magazine) 
Turning (turning) 
*.TOL (tool files, geometry-data)
Default.MGZ (default-magazine) 
 

Together with the installation of the software nccad you will get some extra files. For milling you get the tool-files for the KOSY basic equipment and a relating magazine-file Default.MGZ, which can be modified by you. For milling you get typical examples, too. 


\section{Simulation}

Menu simulation
Revised: August 2002

The machining is shown graphically in different projections.
At the same time, the state of the relays is displayed.
In the menu Parameter/Simulation the different settings can be made.


Choose one of the presentations, and a controlling program is generated, containing the run and controlling the simulation.



Table
The entire X-Y-Z movements are represented as it was set in the menu Parameter/Machine/KOSY under...formats. If the moves set there are smaller than needed for the simulation you get the error message "Unvalid movement..."
You get an overview of how the working piece is machined, but you can't recognize details.

Zoom automatic
The display size of the drawing detail is chosen automatically in order to give a good view of the details.

Zoom detail
The detail shown in the simulation is to be marked by a frame (2 corner points) via the ICON View/Choose detail. 
During the simulation not all of the machining is visible, but the machining of the detail is shown in bigger size.

Table and 3D-view
After the 3-table projection has run as under "table", you can see a 3D-presentation which can be rotated, shifted, enlargened and examined.
Single-step: As soon as the machining has been simulated once, you have the complete controlling program in the cache memory of your PC at your disposal.You can repeat the simulation in different ways in the 3D-window to watch closely. Under "Display" in this 3D-window you can also hide a part of the presentation and choose a single-step presentation. (Switch forward/backward by the keys + and -). 
Table and NC-output
After the 3-table projection has run as under "table", the controlling program can be saved as CNC-program and be used again. As a rule, the CNC-code is adapted to KOSY, an adaption to other CNC-machines is possible on request. See also CNC-export.
This simulation run does not depend on the simulation speed. 
Open GL...
This way of simulation is possible for milling actions and turning actions. It is explained in detail in the relating chapters:
Milling simulation with Open GL or Turning simulation with Open GL.

For this sort of simulation you need a fast PC with very good graphic equipment!! 


\chapter{The Icons}

The icons in the icon menu
Revised: August 2002

In the pictures below you see the ICON menu differently opened. 
After clicking at CAD/CAM.. (..New drawing or ...Open drawing) the Icon menu appears at the left border of the drawing area.


The icons are sorted into groups. When you click at a group-name, the relating icons become accessible. Clicking a second time closes the group again. So the icon bar becomes more clear and structured.

When you move the mouse cursor to an icon and wait for an instant, you can see the tooltip explaining the meaning of it. By pressing the function key <F1>, you 'll get to the relating helppage. 

If you want to get more detailed information about the icons, click at one of the headwords in the help-searchtree or in the index register.


The icon bar after starting.


Example for a tooltip.



 
The upper part of the icon bar with its opened groups.
 
The middle part of the icon bar with its opened groups.
 
The lower part of the icon bar with its opened groups..
 
\section{Edit}
\section{CAD -- 3D}
\section{CAD -- Special}
\section{CAD -- Standard}
\section{CAM -- Standard}
\section{View}
\section{Documentation}
\section{Settings}
\section{Information}
\section{Symbols}
\section{Convert}

\part{Basics CAD}

CAD Basics

Picture: A construction example

Open the branch and select your topic! 


\part{CNC-milling machine KOSY3}

CNC - Milling machines with 3 axles

Picture: Linear profile of the milling machine

Open the branch and select your topic! 

\part{CNC-Driliing machines}

CNC-Drilling machine
 
Picture: A CNC-machine with joystick

Open the branch and select your topic! 

\part{CNC -- Lathe}

CNC-Lathe

\fig{pic/cnclathe/01.png}{Picture: Desktop-Buttons for the different types}

\nccad\ is very versatile; you can programme and operate different types of
machines.
Normally, you get the relating start buttons on your screen after installing the programme. 
After double-clicking a button, \nccad\ is being started with the
necessary range of function.

You have decided on the CNC-lathe so the button on the downright is the right one.

For further help, open the branch and select the topic! 

\chapter{Data and Facts}

The"Intelligent Lathe"
 
Data and Facts

Date: 24 November 2003

Revised: 21 October 2004 

With the KOSY-control and a highly developed version 7 of nccad , the good old lathe 
becomes a very comfortable CNC-Lathe. Let's give you some main information.

\section{1. General remarks about lathes}

The big range of products makes it difficult to choose and invites to make unsuitable comparisons. 
It is necessary to be an expert in order to be able to give reliable judgements on this sort of machines. 
We want to explain you some basic principles.

\subsection{1.1 Main features and types}

The parameters for tip width, tip height and spindle boring have an impact on which worpkieces 
can be machined, whereas the \textbf{following parameters} have also an
\textbf{influence on the quality of the machining}:

\begin{itemize}
\item Driving power 
\item Motor spindle rpm 
\item Accuracy of truth 
\item Accuracy of the feed spindles 
\item Guides and bearings
\end{itemize} 

Normally we distinguish between the following \textbf{types of lathes}:

\begin{itemize}
\item \textbf{Conventional lathes}\\
Movements are made manually by a hand crank or automatically by a mecanical feed drive. 
\item \textbf{CNC-Lathes}\\
Movements and machining are electronically controlled and are prommable through a software. 
\item \textbf{Upgraded conventional lathes}\\
They are extended by an electronic control including a programming/operating software thus becoming a CNC-machine.
Without information and advice, your \textbf{expectations may be disappointed.
You cannot always reach the level of an Original CNC Lathe}.
\end{itemize}

\textbf{Special components are crucial} for different applications, 
\textbf{the most important being}:
\begin{itemize}
\item thread cutting 
\item right-hand/left-hand rotation 
\item speed control 
\item possibility for extension (3rd axis, tool-change, cooling system, etc.)
\end{itemize} 

\subsection{1.2 Problems when conventional lathes are upgraded to become
CNC-lathes}

Often, conventional machines are not constructed for upgrading, therefore:
 
\begin{itemize}

\item \textbf{there are missing suitable space for the drives}\\
e.g. the motors block the way, the connection to the feed spindles is incomplete (type of coupling).

\item \textbf{there are missing helpful construction details}\\
e.g. the turning tool is not placed behind, but in front of the spindle axis. The cuttings are 
difficult to remove in automatic operation and/or covers are inadequate.

\item \textbf{there are missing well-suited feed spindles}\\
Rolled ball screws are a rule in CNC-machines, building them in later is often impossible or the 
chosen spindles are not sufficient (too weak, too big linearity tolerance, wrong bearings, etc.). 
It is best to have them build in by the manufacturer. Conventional lathes have,
as a rule, \textbf{trapezoidal screws with a reverse play}. They are \textbf{not
suitable for CNC-lathes}. In our KOSY-control the reverse play is taken into
account where technically and physically useful.
\begin{itemize}
  \item \textbf{the software can calculate and take into account the reverse
  play only with some applications as}:
  \begin{itemize}
\item thread cutting 
\item plane milling 
\item chipping 
\item cutting 
\end{itemize}
\item \textbf{the following applications are not possible with reverse play}, or
only with considerable faults:
\begin{itemize}
\item contour turning\\
the floating of the supports always leads to mistakes when changing direction, their degree depends 
on the machining conditions (chipping pressure, tool geometry, feed, depth per step, etc.), the leads 
and the measures of the supports. A control without integrated measure system can by no means compensate it. 
\item high precision turning
\end{itemize}
\end{itemize} 

\item \textbf{there are missing important equipment and control elements}\\

The upgrading to a CNC-lathe intends to improve reproducibility and time expenditure and, above all, 
to \textbf{automate the turning procedure}, so that the user has no longer to
keep staying with the machine.
\textbf{For a complete automatic working} the following elements are
\textbf{absolutely necessary}:

\begin{itemize}
\item Process-synchronous control of the turning spindle (ON/OFF and turning
direction) 
\item Speed control of the turning spindle (via control tension 0...10V) 
\item Automatic tool changer\\
No other type of CNC-machines must change the tools as often as CNC-lathes. 
\item Encoder for thread cutting\\
It must be directly connected to the turning spindle (transmission 1:1) and is necessary for 
synchronization of speed and feed. 
\item An integrated controllable cooling system
\end{itemize}

\item \textbf{there are missing consequent new strategies of the users}\\
The hand cranks stay at the machine, for example, although they are completely unnecessary. This may 
lead to mistakes and makes it more difficult to get used to "think in programming level".

\end{itemize}
 
In spite of all these critical words, the increase of performance and comfort shall not be forgotten. 
The investment is worth its money, but, as always, you should think first before acting. This chapter 

\section{2. Special parameters and functions of CNC-lathes with KOSY-control}

The KOSY-control consists of the following elements:

Software nccad 
Control electronics 
Teachware in the form of helppages 
When you want to use the CNC-lathe you need the following parameters and features:

2.1 The Software nccad7 for CNC-lathes
CAD/CAM with direct CNC-control (DNC) for the Coordinate-System KOSY and for foreign machines
The data are valid for the new Version 7,  date of January 2004.
Features in yellow colour will be achieved in the course of autumn/winter 2004/2005.
New Functions of the Version 6/7 with regard to Version 5/6 are marked by + and in colour.

General remarks 

Extensive program for computer assisted drawing (CAD), as well as for programmed machining with CNC-machines (CAM) and for direct machine control (DNC). 
Delivered as multiple-licence on CD, it can be installed as often as needed in your company, institute, school, etc. 
Suitable for Win98/2000/NT/XP. 
Suitable for the installation in networks; pay attention to our tips in the Internet. 
Program needs approx. 60 MB on the harddisk. 
Symbol-libraries included in the delivery. 
+ about 10 example files, examples for beginners of turning. 
Manual (chapter 0, 11) for installation and putting into operation in the CD-box 
System support also via Hotline and Internet 
System-requirements 

As control computer directly at the machine, min.: PC 586 (Pentium1)/min. 133 MHz, min. 16MB RAM, Win98/2000/NT4, CD-drive, graphic resolution min. 800x600, 60 MB free on HD, MS Internet-Explorer V4.02 (or later) must be installed. An old computer can be used here. 
At your constructing- and programming place, min.: PC Pentium2/min. 600 MHz, min. 64MB RAM, Win98/2000/NT/XP, CD-drive, graphic resolution min. 1024x768, 60 MB free on HD, MS Internet-Explorer V4.02 (or later) must be installed.
+ OpenGL with graphic chip set Gforce4 
Operation/Handling 

Modern Windows-user surface with ToolTip-Texts. 
Aktivate the lathe through the menu ("Parameters" and "Machine") 
+ Icon-bar divided into groups, the groups may be opened/closed . 
+ Possibility to make your own Icon-bar to be saved. 
+ Possibility to reduce the range of functions- and operations e.g. for manufacturing personnel or in education. 
Extremely extended HTML-Helpsystem with searchtree and full text search + Index register, 
with INTERNET-updating + direct access via F1-key. 
CNC-Teil (Functions with special impact for turning)

Machine control together with the KOSY- control electronics
- through microcontroller independent from computer speed
+ use of Spooler for workpiece machining without interruptions
- automatic HOME- positioning (end switches)
- go to CHANGE WORKPIECE -POSITION by pressing button
- hand-control of all axes in 5 different speeds
- display of X-Y-Z-position with a resolution of 1/100mm
- linear interpolation in 3 axes (linear movement in space),
- circular interpolation alternatively in 2 of 3 axes (circular movement X/Y- or X/Z or Y/Z-level)
- enter direct orders with repeat function
- direct enter in relative- or absolute mode
- single-step-working
- interruption/continuing of working (f.e.after breaking a tool)
- workpiece copies in X-and Y-direction
- memory for 20 diff. workpiece zero-points
- direct start of CNC-Programs when booting the PC
- texts for digital inputs freely editable
- testfunction for digital inputs
- support of automatic tool change
- measuring functions for turning tools
- free parameters of machine data for any mechanics
+ remote monitoring of the machining state via network (TCP/IP). Instructions on demand ! 
Scanning
- taking in of the X-Y-Table position as CNC-command (TeachIN)
- take in the table position as coordinate pair in CAD 
- scanning the height for general measurement science, the series of measurements is written as a evaluable file 
CNC-Programmiung
-Code sentences after DIN/ISO 66025
-approx. 30 different NC-commands
-support of relays and digital inputs
-support of analogue inputs/outputs
-support of an 3rd axis (Option) 
-syntax-check
-teach-in-mode with automatic command generating (including relay)
-import of NC-files as connection to 3D-SW- and postprocessors (external CAD/CAM)
-graphic simulation of NC-programs 
Text-Editor
-text extent to approx. 500.000 lines, 80 signs/line
-full-screen editing with horizontal and vertical scroll
-line- and column display, help-function
-block-operations: delete, shift, copy
-file-functions: print, save, save as, load)
-ASCII-import (NC-program was made in another text editor) 
 

Special turning funtions 

Make operation easier
- Manual user surface for conventional turning (manual command by mouse and keyboard)
- Tool administration with correction memory (X,Z-Position, + geometry), manual shift through keyboard (Tn)
+ Editing of tool geometry (tool description)
+ Editing tool sets in magazines of frequently used tools
- Machine related tool change point
- Machine related workpiece-changing position
- Stop function for clamping in 
Special turning cycles and turning functions (in CAD/CAM and CNC )
- Length - thread cutting (external, internal, left, right, norm-thread, inch-thread, individual linear thread)
- Plane-thread cutting (Spirals)
- Cone- thread cutting (clamp thread)
- More lead thread cutting
- Thread repairing (insertion in existing threads)
- Chipping cycles (longside, plane, inside, outside, from or from right)
- Boring cycles for centric borings (with/without chip-break or de-chipping)
- Boring threadswith compensation chuck
- Cutting cycles (left, right, outside, with/without edge chanmfer)
- Sting cycles (symmetrical/asymmetrical, outside, inside, with/without edge chamfer) 
- Blade radius-correction (account for tool geometry)
- Cone turning
- Radius function (saves special tools and polishing)
- account for the reverse play (as far as possible, not with contour turning, high precision turning) 
Helps
- Simulation of turning in CAD/CAM + and CNC 
+ OpenGL-simulation with viewing of tool geometry
+ OpenGL-Simulation of drilling and interior turning 
- Workshop programming without CNC-knowledge 
CAD-part (only drawing functions with general use for turning) 

Draw
- Drawing turning machining in different layers and levels (auxiliary turning axes)
-Input of polar coordinates or cartesian coordinates absolute and relative, 
+ calculating input (4 basic calculations with parentheses).
+drawing by CNC-Machine (the X-Y-position of the machine is taken in the drawing area by pressing a button). 
Drawing elements
- Reference point,  line,  line with indication of angles, polygon, arch, ellipse,
- curve (approximation, interpolation, math.formula, free hand), 
- workpiece-zeropoint, 
- chamfering and rounding, selective chamfering and rounding
+ cutting, stinging, threads
+ drilling 
View
- Presentation on table (DIN A4), detail (also with sectional view)
- moving the detail (also while drawing and with positioning window)
- new presentation
+ Scrolling detail with scroll-bar or cursor keys, centering by mouse and Enter-key 
Edit
- Delete (window, separate, last, last part of polygons or curves)
- shift (separate, window), copy (separate, window)
- mirror hor./vert (with/without copy, also text)
- change features, trim
- Undo (1 step)
- shift construction-point
- contour correction
- follow contour (autom. trimming and closing)
- thin out contours
- dissolve/dismantle contours into lines or polygons
+ Split up and drop a perpendicular with lines. 
+ Scale drawing elements and groups. 
Symbols
- Save, load, shift, copy, dissolve, scale (autom. scale)
- libraries for electrics, electronics, pneumatics, mechanics, architecture, logos and accessories
+ ICONs as drawing elements for making teachware. 
Elements for documentation
- Dimensions (horizontal, vertical, angles, radius, sloping), 
+ editable, font height to be set.
- Inscriptions with text editor. 
+ drawing elements for set documentation texts: Date, file name, worker. 
Information-elements
+ Measuring distances
+ Information about chosen drawing elements with calculated data (area, circumference, number, etc.). 
Layers
Drawing in max. 10 layers (8 freely eligible) ,automatic color switching, button "display all or hide" 
+ layer protection (no editing/changing of the drawing possible) 
Parameters/settings
-Ruler, catch, grid, line diameter, line type, colour assignation
- scale (screen-nature)
- drawing-area, Y -table size
- help lines
- default technology. 
File-operations
Load, save, save as. 
Print/Plot
According to Windows installation, printing of help pages via Internet-Browser.
True to scale printing with half-automatic scale calculation and scaling.
+ Print with set paper format DIN A4.
+ Print with freely eligible scale factor.
Plot via HPGL-files. 
Import
DXF-Import (DXF = exchange format for 2D-drawings) + improved and extended. 
HPGL-Import (Plotter-files). 
Export
- Port to DTP via HPGL-files.
- Export of control-list (attach the KOSY-list to CAD-files). 
- CNC-Program for KOSY-control, for other CNC-lathes (Option)
+ Save the drawing as Bitmap (BMP) in cache memory by pressing the keys Ctrl+C. 
2.2 Special elements of KOSY-control electronics for CNC-lathes
The electronis, derivated from our universal KOSY-control, contains the following elements, specifically for lathes:

Stepping motor-Interface 

Endsteps
- For 2-phase-stepping motors up to 4 A phase current, normal equipment for 2 stepping motors
- optionally you can use foreign endsteps (Signal output for frequency, direction and current reduction at standstill)
- Third endstep e.g. for tool-changing system possible (optional) 
Box
- Microcontroller, Interface and power supply in a closed box
- Plug connections for stepping motors, step by step transmitter (optional), PC and control functions 
Control elements 

Relay-outputs
- 1 output for turning spindle (ON/OFF) 
- 4 more relays (switching contact), switch performance 24V/1A optional
-- (e.g. for truning direction, the control electronics of the turning spindle must be suitable for remote control) 
Digital inputs (optional)
- 5 Optocoupler-inputs, 24V, for compulsory sequence control 
Tension output (optional)
- 1....24V (1...10V) to control the number of revs of the turning spindle 
step be step transmitter-input
- Plug connection and connection to microcontroller 
 


\part{Dispensing systems}

Dispensing systems

 
Picture: Example for a dispensing system

Open the branch and select your topic! 

\part{Automation systems}

Automating Systems

Picture: An example for automating: dip varnishing

Open the branch and select your topic! 

\part{KOSY Control units}

KOSY Controls
 
Picture: KOSY Control 4.7 in standard version

Open the branch and select your topic! 

\part{Special functions/programmes}

Help programmes


Pictures: Fonts designed by the help programme zse

Open the branch and select your topic! 


\part{Import/Export}

Import/Export
 
Picture: File import in nccad

Open the branch and select your topic! 

\part{Options and their operation}

This is a generic term



Please continue searching between the indented entries. 

\part{How to handle the system}

This is a generic term



Please continue searching between the indented entries. 

\part{Appendix}

This is a generic term



Please continue searching between the indented entries. 

\end{document}
\documentclass[12pt,a4paper,oneside]{book}
\usepackage[a4paper,left=20mm,right=10mm,top=10mm,bottom=10mm]{geometry}
%   bindingoffset=0mm}

\usepackage[utf8]{inputenc}
\usepackage[german,english,russian]{babel}
\usepackage[unicode]{hyperref}
% \hypersetup{unicode=true}

\usepackage{graphicx}
\usepackage{xcolor}
% \usepackage{menukeys} can't download package
\newcommand{\keys}[1]{\textbf{[#1]}}
\newcommand{\file}[1]{\textbf{#1}}

\newcommand{\DE}[1]{\textcolor{green}{#1}} 
\newcommand{\RU}[1]{\textcolor{red}{#1}}
\newcommand{\TRpart}[2]{\part{#2 /#1/}}
\newcommand{\TRchapter}[2]{\chapter{#1\\#2}}
\newcommand{\TRsection}[2]{\section{#1\\#2}}
\newcommand{\TRsubsection}[2]{\subsection{#1\\#2}}
\newcommand{\TRsubsubsection}[2]{\subsubsection{#1\\#2}}
\newcommand{\TRparagraph}[2]{\paragraph{#1\\#2\\}}



\title{NCCAD7: Основы САПР}
\author{перевод Dmitry Ponyatov <dponyatov@gmail.com>}

\begin{document}

\maketitle
\tableofcontents
\bigskip

\TRparagraph{Grundlagen CAD}{Основы САПР}
\TRparagraph{CAD - Konstruktionsgrundlagen}{САПР - Основы дизайна}
\TRparagraph{Allgemeines}{Общая информация}

\DE{Um ein CAD-Programm zu verstehen, muß man sich zunächst mit dem
Bedienprinzip auseinandersetzen.}
\RU{Чтобы понять САПР-программу, нужно сначала разобраться с принципами работы.}
\DE{Es gibt beim Vergleich der verschiedenen
CAD-Programme zwangsläufig Gemeinsamkeiten, wie beispielsweise die Tatsache,
daß geometrische
Grundfiguren ausgewählt werden müssen.}
\RU{Сравнивая различные САПР, многие черты должны быть общими, например, тот
факт, что геометрические примитивы должны быть выбраны.}
\DE{Es gibt aber auch Unterschiede, wie
beispielsweise die Belegung der rechten Maustaste.}
\RU{Но есть и отличия, например, назначение правой кнопки мыши.}
\DE{Sie sollten deshalb die folgenden Punkte durcharbeiten und in der praktischen
Arbeit dieses Wissen anwenden.}
\RU{Таким образом, вы должны работать используя описанные далее приемы, и
применять эти знания в практической работе.}
\DE{Entweder Sie gehen in der Reihenfolge der Nummerierung alle Punkte durch
oder Sie klicken auf eines der Themen im folgenden Inhaltsverzeichnis, um zur
gewünschten Erklärung zu kommen.}
\RU{Либо вы идете в порядке нумерации всех терминов или нажмите на одну из тем в
следующей таблице содержания, чтобы получить требуемое.}
\DE{Zum Inhaltsverzeichnis kehren Sie immer wieder zurück, wenn Sie in eines der
Felder [Zurück] - oder im Browser-Menü den Button "Zurück" klicken.}
\RU{Чтобы вернуться к содержанию, если вы находитесь в одном из полей
\keys{назад} - кнопка \keys{Назад} или нажмите в меню обозревателя.}

% \TRsubsection{Inhaltsverzeichnis der Themen}{Индекс темы}

\TRchapter{Start}{Начало}

\includegraphics{pic/Kons1.png}\bigskip

\DE{Zunächst müssen Sie im Menü Datei entscheiden, ob Sie mit CNC oder CAD arbeiten
wollen, eine neue Datei erstellen- oder eine vorhandene laden wollen.}
\RU{Во-первых, вы должны решить выбрав в меню Файл, хотите ли вы работать с ЧПУ
или САПР, создать новый файл или загрузить существующий.}
\DE{Bei CNC
gelangen Sie in den Texteditor, um CNC-Programme erstellen, bearbeiten oder
ausführen zu können.}
\RU{В режиме ЧПУ откроется текстовый редактор для создания программы ЧПУ, чтобы
редактировать ее, с возможностью выполнения.}
\DE{Bei CAD starten Sie ein spezielles Konstruktionsprogramm
mit optimaler Verbindung zur CNC-Maschine.}
\RU{При выборе режима САПР запускается программа конструирования, с оптимальной
адаптацией для использования совместно со ЧПУ-станком.}
\DE{Um eine neue Zeichnung konstruieren zu können klicken Sie auf \keys{CAD/CAM
- Neue Zeichnung}.}
\RU{Для того чтобы построить новый чертеж, нажмите на \keys{CAD / CAM - новый
рисунок}.}

\TRchapter{Der CAD-Bildschirm}{Экран САПР}

\includegraphics{pic/Kons2.png}\bigskip

\DE{Sobald Sie im Menü \keys{Datei}/\keys{CAD/CAM - Neue Zeichnung} gewählt
haben, können Sie im Zeichenfeld konstruieren.}
\RU{После того как вы выбрали в меню \keys{Файл}/\keys{CAD/CAM Новый рисунок},
можно построить характер поля.}
\DE{Im Bild oben wird Ihnen die Bedeutung der
einzelnen Bereiche des CAD-Bildschirms erklärt.}
\RU{На картинке выше вы видите несколько важных областей экрана САПР.}
\DE{Links neben dem Zeichenfeld ist das ICON-Menü.}
\RU{В левой части области рисования расположено иконное меню.}
\DE{In ihm können die wichtigsten Funktionen direkt gewählt werden.}
\RU{Наиболее важные функции могут быть выбраны непосредственно в нем.}

\TRsection{Bereiche der ICON-Hauptleiste}{Области основной
    панели инструментов}

Иконное меню разделено сверху вниз на разные группы:

\begin{itemize}
  \item
\DE{Der Technologie-Bereich, der den Zugang zu den CAM- (Bearbeitungs-)
Funktionen eröffnet.}\\
\RU{Область технологий, открываеющая доступ к CAM-функциям обработки.}
  \item
\DE{Der Einstell- und Darstellungsbereich für die die Auswahl der Layer, Linien
usw. - und für die Behandlung des Zeichnungs-Ausschnittes.}\\
\RU{Область установки и отображения для выбора слоя, линии, и т.д. - а также для
лечения подписки вырез.}
  \item
\DE{Der Korrekturbereich zum Bearbeiten (editieren) von Zeichnungsteilen. }\\
\RU{Редактирование областей компенсации для обработки деталей согласно чертежа.}
  \item
\DE{Der 2D-Bereich für die Konstruktion von Zeichnungsteilen in der 
X-Y-Ebene.}\\
\RU{2D области для проектирования элементов чертежа в плоскости XY.}
  \item
\DE{Der 3D-Bereich für die Konstruktion von "Plastischen Zonen". }\\
\RU{3D область для задания "пластических зон".}
  \item
\DE{Der Dokumentations- und Informationsbereich }\\
\RU{Область документирования и информации}
\end{itemize}

\DE{Bewegen Sie zum Kennenlernen der Funktionen die Maus vom Zeichenfeld zum
ICON-Menü und dort zu einem beliebigen ICON (beispielsweise zu Gerade im 2D-Bereich),
warten Sie einen Augenblick.}
\RU{Переместите указатель мыши на любой значок в иконном меню, чтобы узнать
о его функции (например в 2D области), подождите пару секунд.}
\DE{Es erscheint ein Erklärungsfeld (ToolTip) mit der
Bedeutung dieses ICONs.}
\RU{Вы увидите всплывающее поле подсказки с функцией этой иконки.}

\TRsection{Die Status-Zeile}{Строка состояния}

\includegraphics{pic/Kons3.png}\bigskip

\DE{Sie hat eine besondere Aufgabe:}
\RU{Она имеет конкретную задачу:}
\DE{Dort "sagt" Ihnen nccad, welche Bedienschritte erwartet werden (z.B.:
Startpunkt einer Geraden wählen). }
\RU{В ней NCCAD "говорит" Вам какая операция будет выполнена (например,
выбор начальной точки линии).}
\DE{Alle zu konstruierenden Zeichnungsteile basieren auf mathematischer
Grundlage, d.h.
es müssen Koordinatenpunkte angegeben werden, mit deren Hilfe die geometrischen
Figuren berechnet - und dargestellt werden können.}
\RU{Все элементы строятся на математической основе, то есть должны быть указаны
координаты, с помощью которых рассчитываются геометрические фигуры - и могут быть отображены.}
\DE{Zunächst ist kein Zeichnungsteil vorgeschlagen, wählen Sie z.B in der}
\RU{Во-первых, ни одна часть чертежа не предлагается выбрать для}
\DE{ICON-Gruppe}
\RU{в группе иконок}
\DE{\keys{CAD Standard}}
\DE{das ICON GERADE.}
\RU{в панели иконок.} 
\DE{Sie benötigt zwei Punkte (Startpunkt und Endpunkt), die durch 2 Mausklicks (kurzes
Drücken der linken Maustaste) an 2 verschiedenen Stellen des Zeichenfeldes mit dem
 Fadenkreuz positioniert werden.}
\RU{Вам нужно указать две точки (начальная точка и конечная точка) сделав 2
щелчка (нажатием левой кнопки мыши), которые будут проставлены в двух разных 
местах рабочей области, указанных символом перекрестия.} 
\DE{Grundsätzlich wird in der Statuszeile angezeigt,
 welcher Konstruktionspunkt momentan eingegeben werden muß.}
\RU{Принцип указывается в строке состояния, в момент ввода конструкционной
точки.}

\TRchapter{CAD-Funktionswahl}{Функции выбора САПР}

\includegraphics{pic/Kons4.png}\bigskip

\TRchapter{Koordinaten beim Konstruieren}{Координаты в конструировании}

\includegraphics{pic/Kons5.png}\bigskip

    \TRsection{Möglichkeiten der Koordinaten-Angabe}{Cпособы
    задания координат}
    \TRsection{Konstruktionsbeispiel}{Пример расчета}
    \TRsection{Die Möglichkeiten der Koordinaten-Eingabe}{Возможности 
    входных координат}

\TRchapter{Ausschnitte und Darstellungsformen}{Разделы и формы}

\includegraphics{pic/Kons6.png}\bigskip

    \TRsection{Unmittelbar Veränderung der Darstellung}{Немедленно
    смените представление}
    \TRsection{ICONS für die Wahl der Darstellungsformen}{Иконки
    для выбора форматов отображения}
    \TRsection{Fang}{Привязка}

\TRchapter{Besondere Konstruktionshilfen}{Особый дизайн СПИДом}

    \TRsection{Maus fixieren (Orthogonales Zeichnen)}{Мышь Fix
    (ортогонального рисования)}
    \TRsection{Fang von Konstruktionspunkten mit dem Suchfenster}{Строительство
    ловить точки с окном поиска}
    \TRsection{Fang von deckungsgleichen Konstruktionspunkten}{привлекательным
    дизайном конгруэнтным пунктов}
    \TRsection{Gruppenbildung}{Группировка}
    \TRsection{Tastaturbelegung}{Макет}

\TRchapter{Layer (Zeichnungslagen)}{Слой (материалы для рисования)}

    \TRsection{Layer für Zeichnungs- und Frästeile}{Слои для рисования и
    фрезерной частей}
    \TRsection{Besondere Layer}{Специальный слой}

% 	\TRsection{CAD - Kurzanleitung}{CAD - краткое руководство}
% 	\TRsection{CAD - 3D Funktionen}{CAD - 3D-функции}
% 	\TRsection{CAD - Technisches Zeichnen}{CAD - Технический рисунок}
% 		\TRsubsection{Dreiseiten-Darstellung}{Три-представление}
% 		\TRsubsection{Räumliche Darstellung}{Пространственное представление}
% 		\TRsubsection{Arbeiten mit Symbolen}{Работа с символами}
% 	\TRsection{Grafik}{Графика}
% 		\TRsubsection{Bilder importieren}{Импорт изображений}
% 		\TRsubsection{Formulare}{Формы}

\end{document}

